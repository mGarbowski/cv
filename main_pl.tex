%-------------------------------------
% Resum in LaTeX (XeLaTeX)
% Author:  HF Yuan
% Project: https://github.com/xyz-yuanhf/yuan-resume
% Base on: https://github.com/mattyHerzig/mattys_resume
%          http://www.jianxu.net/en/
% License: MIT
%------------------------------------

\documentclass[a4paper, 10pt]{article}
\usepackage[polish]{babel}
\usepackage[T1]{fontenc}
\usepackage{myresume}  % Pakiet stylu


% --------------------  START  --------------------
\begin{document}

% -------------------- NAGŁÓWEK --------------------
\begin{flushright}
  \setstretch{0.6}
  \item {\Calluna (+48) 665522039}  % Numer telefonu
  \item {\Calluna mikolaj.garbowski@gmail.com}  % Adres e-mail
  \item {\Calluna https://github.com/mGarbowski}  % Strona domowa
\end{flushright}\vspace{-45pt}

\begin{flushleft}
  {\Calluna \fontsize{30pt}{30pt}\selectfont \textsc{Mikołaj Garbowski}}
  \noindent\rule{\textwidth}{0.4pt}
\end{flushleft}

% -------------------- EDUKACJA --------------------
\sectionBlock{
\section{Edukacja}
}{
\eduHeading
  {Wydział Elektroniki i Technik Informacyjnych}{Politechnika Warszawska}
  {Informatyka, inżynierskie}{2022 - 2026 \footnotesize{\textit{(planowane)}}}
\itemListStart
  \myItem{Specjalizacja: Sztuczna Inteligencja}
  \myItem{W programie stypendialnym \textit{studia id} 2023-2025}
  \myItem{Temat pracy inżynierskiej: \textit{Poprawa jakości systemu rozpoznawania twarzy w warunkach rzeczywistych przez augmentację danych treningowych}}
\itemListEnd

}

% -------------------- DOŚWIADCZENIE --------------------
\sectionBlock{
\section{Doświadczenie}
}{
\internHeading
  {Hermes Data \& Software Solutions}{Warszawa, Polska}{03.2025 - obecnie}
\itemListStart
  \myItem{Fullstack Developer}
  \myItem{Implementacja systemu CRM dla centrum biznesowego}
  \myItem{TypeScript, React, Python, Django REST Framework, PostgreSQL}
\itemListEnd
\internHeading
  {Ringier Axel Springer Polska}{Warszawa, Polska}{07.2024 - 09.2024}
\itemListStart
  \myItem{Fullstack Developer (stażysta)}
  \myItem{Stworzenie produkcyjnej aplikacji webowej do edycji zdjęć}
  \myItem{TypeScript, NodeJS, React, AWS}
\itemListEnd
\internHeading
  {Sages}{Warszawa, Polska}{07.2021 - 09.2021}
\itemListStart
  \myItem{Junior Support Engineer}
  \myItem{Praca nad wdrażaniem systemu Omega-PSIR dla instytucji akademickich}
\itemListEnd
}

% -------------------- UMIEJĘTNOŚCI --------------------
\sectionBlock{
\section{Umiejętności}
}{
\skillListStart
\justifying
\item \emph{Języki obce}: angielski C2, francuski A2.
\item \emph{Python}: 
    Stos naukowy i ML (
      \tech{numpy},
      \tech{scikit-learn},
      \tech{PyTorch},
      \tech{transformers},
      \tech{Weights \& Biases}
    ),
    Web (
      \tech{Django},
      \tech{Django REST Framework},
      \tech{FastAPI}
    ).
\item \emph{TypeScript/JavaScript}: 
    \tech{React}, 
    \tech{Redux}, 
    \tech{NodeJS}.
\item \emph{Java}: 
    \tech{Spring Framework}.
\item \emph{Inne języki}: 
    \tech{C++}, 
    \tech{Rust}.
\item \emph{Bazy danych SQL i NoSQL}:
    \tech{PostgreSQL},
    \tech{ChromaDB}.
\item \emph{DevOps}: 
    \tech{Git}, 
    \tech{Docker}, 
    \tech{CI/CD}, 
    \tech{Terraform}, 
    \tech{Ansible}, 
    \tech{Azure}, 
    \tech{Linux}, 
    \tech{Bash}.
\item \emph{Inne}: Uczenie maszynowe, głębokie sieci neuronowe, modele językowe, NLP, algorytmy optymalizacji, architektura i inżynieria oprogramowania, modelowanie (\tech{UML}, \tech{BPMN}).
\skillListEnd
}

% -------------------- PROJEKTY --------------------
\sectionBlock{
\section{Projekty}
}{

\projHeading
{\textbf{Olivia CRM}}
{Implementacja i rozwój systemu CRM dla centrum biznesowego. 
\newline TypeScript, React, Python, Django/DRF, PostgreSQL}
{Hermes Data \& Software}

\projHeading
{\textbf{Foto Editor}}
{Front-endowa aplikacja do edycji zdjęć oparta na Canvas API. Zintegrowana z wewnętrznym systemem wydawniczym. 
\newline TypeScript, React, Redux, FabricJS.}
{RASP}

\projHeading
{\textbf{Badania nad modelami rozpoznawania twarzy} \githubIcon{https://github.com/mGarbowski/thesis}}
{Badania nad technikami poprawy jakości systemu opartego na splotowych sieciach neuronowych, 
z wykorzystaniem rzeczywistych danych. Implementacja własnego systemu rozpoznawania twarzy.
\newline Python, PyTorch, OpenCV, Dlib, Weights \& Biases.}
{Praca inżynierska, PW}

\projHeading
{\textbf{Mikroserwis z modelem predykcyjnym dla klona Booking.com} \githubIcon{https://github.com/mGarbowski/ium-projekt}}
{Analiza danych, inżynieria cech, wytrenowanie modelu predykcyjnego i implementacja mikroserwisu wykonującego predykcję oceny hotelu.
\newline Python, PyTorch, Scikit-learn, Pandas, FastAPI, Docker.}
{Projekt grupowy, PW}

\projHeading
{\textbf{Aplikacja RAG dla zbioru notatek z wykładów} \githubIcon{https://github.com/mGarbowski/llm-projekt}}
{Czatbot oparty na modelu językowym Bielik, wykorzystujący podejście Retrieval-Augmented Generation.
\newline Python, PyTorch, HuggingFace, sentence-transformers, ChromaDB, FastAPI.}
{Projekt indywidualny, PW}

\projHeading
{\textbf{Ewolucja i symulacja pojazdów} \githubIcon{https://github.com/mGarbowski/zpr-projekt}}
{Aplikacja desktopowa z silnikiem fizycznym i algorytmem ewolucyjnym. 
\newline C++, ImGui, SFML, Box2D.}
{Projekt grupowy, PW}

\projHeading
{\textbf{Renderowanie roślin z użyciem L-systemów w OpenGL} \githubIcon{https://github.com/mbienkowsk/LSystemTreeGenerator}}
{Aplikacja desktopowa z GUI, renderer sceny 3D.
\newline OpenGL, Rust}
{Projekt grupowy, PW}

\projHeading
{\textbf{Wdrożenie aplikacji webowej na chmurze publicznej} \githubIcon{https://github.com/mGarbowski/wus-lab-2}}
{Zautomatyzowane wdrożenie aplikacji na chmurze Azure.
\newline Azure, Ansible, Terraform, Docker, Bash.}
{Projekt grupowy, PW}

% \projHeading
% {\textbf{Optymalizacja ruchu w sieci telekomunikacyjnej} \githubIcon{https://github.com/mGarbowski/pzsp2-projekt}}
% {Model programowania całkowitoliczbowego oraz aplikacja webowa.
% \newline Pyomo, FastAPI, React, Docker.}
% {Projekt grupowy, PW}

% \projHeading
% {\textbf{Platforma do recenzowania kursów językowych} \githubIcon{https://github.com/threescomplement/pap-projekt}}
% {Aplikacja webowa. 
% \newline Spring, React, PostgreSQL, Docker.}
% {Projekt grupowy, PW}

}

\vfill
\noindent\small Wyrażam zgodę na przetwarzanie moich danych osobowych zawartych w mojej aplikacji dla potrzeb niezbędnych do realizacji procesu rekrutacji zgodnie z ustawą z dnia 10 maja 2018 r. o ochronie danych osobowych (Dz. U. 2018, poz. 1000) oraz Rozporządzeniem Parlamentu Europejskiego i Rady (UE) 2016/679 z dnia 27 kwietnia 2016 r. w sprawie ochrony osób fizycznych w związku z przetwarzaniem danych osobowych (RODO).

\end{document}
