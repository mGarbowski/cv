%-------------------------------------
% Resum in LaTeX (XeLaTeX)
% Author:  HF Yuan
% Project: https://github.com/xyz-yuanhf/yuan-resume
% Base on: https://github.com/mattyHerzig/mattys_resume
%          http://www.jianxu.net/en/
% License: MIT
%------------------------------------

\documentclass[a4paper, 10pt]{article}
\usepackage[polish]{babel}
\usepackage[T1]{fontenc}
\usepackage{myresume}  % Pakiet stylu


% --------------------  START  --------------------
\begin{document}

% -------------------- NAGŁÓWEK --------------------
\begin{flushright}
  \setstretch{0.6}
  \item {\Calluna (+48) 665522039}  % Numer telefonu
  \item {\Calluna mikolaj.garbowski@gmail.com}  % Adres e-mail
  \item {\Calluna https://github.com/mGarbowski}  % Strona domowa
\end{flushright}\vspace{-45pt}

\begin{flushleft}
  {\Calluna \fontsize{30pt}{30pt}\selectfont \textsc{Mikołaj Garbowski}}
  \noindent\rule{\textwidth}{0.4pt}
\end{flushleft}

% -------------------- EDUKACJA --------------------
\sectionBlock{
\section{Edukacja}
}{
\eduHeading
  {Wydział Elektroniki i Technik Informacyjnych}{Politechnika Warszawska}
  {Informatyka, inżynierskie}{2022 - 2026 \footnotesize{\textit{(oczekiwane)}}}
\itemListStart
  \myItem{Specjalizacja: Sztuczna Inteligencja}
  \myItem{Od 2023 roku w programie stypendialnym \textit{studia id}}
\itemListEnd

\eduHeading
  {XXVII LO im. Tadeusza Czackiego}{Warszawa, Polska}
  {Profil matematyczno-fizyczno-informatyczny}{2019 - 2022}
}

% -------------------- STAŻE --------------------
\sectionBlock{
\section{Doświadczenie}
}{
\internHeading
  {Ringier Axel Springer Polska}{Warszawa, Polska}{07.2024 - 09.2024}
\itemListStart
  \myItem{Fullstack Developer (stażysta)}
  \myItem{Stworzenie produkcyjnej aplikacji webowej do edycji zdjęć}
  \myItem{TypeScript, NodeJS, React, AWS}
\itemListEnd
\internHeading
  {Sages}{Warszawa, Polska}{07.2021 - 09.2021}
\itemListStart
  \myItem{Junior Support Engineer}
  \myItem{Praca nad wdrażaniem systemu Omega-PSIR dla instytucji akademickich}
\itemListEnd
}

% -------------------- UMIEJĘTNOŚCI --------------------
\sectionBlock{
\section{Umiejętności}
}{
\skillListStart
\justifying
\item \emph{Języki}: angielski C2, francuski A2.
\item \emph{Programowanie}: 
    {\Courier \fontsize{11pt}{11pt}\selectfont Python},
    {\Courier \fontsize{11pt}{11pt}\selectfont TypeScript},
    {\Courier \fontsize{11pt}{11pt}\selectfont Java}, 
    {\Courier \fontsize{11pt}{11pt}\selectfont C++}, 
    {\Courier \fontsize{11pt}{11pt}\selectfont SQL},
    {\Courier \fontsize{11pt}{11pt}\selectfont React},
    {\Courier \fontsize{11pt}{11pt}\selectfont Spring}.
\item \emph{Narzędzia}:
    {\Courier \fontsize{11pt}{11pt}\selectfont Git},
    {\Courier \fontsize{11pt}{11pt}\selectfont GitHub Actions},
    {\Courier \fontsize{11pt}{11pt}\selectfont Docker},
    {\Courier \fontsize{11pt}{11pt}\selectfont Linux},
    {\Courier \fontsize{11pt}{11pt}\selectfont Bash}.
\item \emph{Inne}: Uczenie maszynowe, optymalizacja, wzorce projektowe, programowanie liniowe, UML.
\skillListEnd
}

% -------------------- PROJEKTY --------------------
\sectionBlock{
\section{Projekty}
}{

\projHeading
{\textbf{Olivia CRM}}
{Implemnetacja systemu CRM}
{Hermes Data \& Software}

\projHeading
{\textbf{Foto Editor}}
{Front-endowa aplikacja do edycji zdjęć oparta na Canvas API. Zintegrowana z wewnętrznym systemem wydawniczym. TypeScript, React, Redux, FabricJS.}
{RASP}

\projHeading
{\textbf{Optymalizacja ruchu w sieci telekomunikacyjnej}}
{Model programowania całkowitoliczbowego oraz aplikacja webowa. Pyomo, FastAPI, React, Docker.}
{Projekt grupowy, PW}

\projHeading
{\textbf{Platforma do recenzowania kursów językowych}}
{Aplikacja webowa. Spring, React, PostgreSQL, Docker.}
{Projekt grupowy, PW}

\projHeading
{\textbf{Ewolucja i symulacja pojazdów}}
{Aplikacja desktopowa implementująca algorytm ewolucyjny do optymalizacji modeli samochodów w symulacji. C++, ImGui, SFML, Box2D.}
{Projekt grupowy, PW}

\projHeading
{\textbf{ID3 i modele zespołowe dla problemu klasyfikacji wieloklasowej}}
{Porównanie klasycznych drzew decyzyjnych ID3 i modeli zespołowych wykorzystujących podejście One-vs-One i One-vs-Rest. Python.}
{Projekt indywidualny, PW}
}

\vfill
\noindent\small Wyrażam zgodę na przetwarzanie moich danych osobowych zawartych w mojej aplikacji dla potrzeb niezbędnych do realizacji procesu rekrutacji zgodnie z ustawą z dnia 10 maja 2018 r. o ochronie danych osobowych (Dz. U. 2018, poz. 1000) oraz Rozporządzeniem Parlamentu Europejskiego i Rady (UE) 2016/679 z dnia 27 kwietnia 2016 r. w sprawie ochrony osób fizycznych w związku z przetwarzaniem danych osobowych (RODO).

\end{document}
